\documentclass{article}
\usepackage[czech]{babel}
\usepackage[utf8]{inputenc}
\usepackage{a4wide}

\title{\Huge{Unixové systémy a shell 2021/2022}\\
\vspace{1em}
\normalsize{Přednášky s trochou omáčky}}
\author{\normalsize Jakub Karel}
\date{\today}

\begin{document}
\maketitle
\begin{abstract}
Tento dokument vzniká jako zcela přepracovaná verze dokumentu, který k podobnému předmětu vznikl před mnoha lety. Cílem je poskytnou sobě a ostatním studentům ucelený text, který bude obsahovat veškeré potřebné informace k úspěšném splnění předmětu a to čtivou formou a s řadou doplňkových informací a ukázek. Pokud se nespokojíte s nutným minimem informací z prezentace z přednášky, pak jste na správném místě. Protože jsem Linuxový pleb bez časového komfortu, dá se předpokládat, že v textu budou překlepy a faktické chyby. Pokud mi pomůžete tyto chyby odstranit, bude mi ctí vaše jméno či přezdívku s velkým díkem zapsat do úvodního slova. Použití textu je pouze na vlastní nebezpečí!!!
\end{abstract}

\thispagestyle{empty}
\newpage
\thispagestyle{empty}
\tableofcontents
\thispagestyle{empty}
\newpage
\setcounter{page}{1}

\section{Úvodní slova}
\subsection{Testovací prostředí}
K vytvoření tohoto dokumentu byl použit typografický nástroj \LaTeXe. K testování příkazu a skriptů jsem použil {\bf Ubuntu} nainstalované na Windows 11 v rámci {\bf Windows Subsystem for Linux}. Pokud pracujete na Windows, mohu tento postup doporučit.

//TODO návod na WSL?

\subsection{Konvence textu}
Všechny texty, příkazy, parametry, skripty\dots Zkrátka vše, co se zadává do terminálu a zároveň všechny odpovědi terminálu, budou vysázeny následujícím způsobem.
\begin{verbatim}
ping google.com
\end{verbatim}
Přestože každý nový řádek v shellu vypadá nějak takto:
\begin{verbatim}
kareja00@DESKTOP-720CJ5H:~$
\end{verbatim}
\dots budu ve většině ukázek tuto část, ač je velice důležitá, pro zjednodušení vynechávat. Parametry představené u některých funkcí rozhodně nejsou všechny parametry, které tyto funkce umí zpracovat. Dokonce i funkce, které jsou představeny jako funkce bez parametrů, jich umí většinou řadu zpracovat. Úroveň hloubky zkoumání tedy odpovídá aktuálním výukovým potřebám. Další informace je možné získat z manuálových stránek.

\subsection{Errata}
Pokud v dokumentu narazíte na chybu, prosím neváhejte a ozvěte se mi.

\newpage






\end{document}